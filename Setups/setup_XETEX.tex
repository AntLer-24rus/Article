%!TEX root = article.tex
\usepackage[noblocks]{authblk}                      % Для добавления ссылок на авторов
% после знаков препинания пробелы не увеличиваются
\frenchspacing

%%% Рисунки в текста %%%
\usepackage{wrapfig}

\RequirePackage{etoolbox}[2015/08/02]               % Для продвинутой проверки разных условий

%%% Поля и разметка страницы %%%
\usepackage{pdflscape}                              % Для включения альбомных страниц
\usepackage{geometry}                               % Для последующего задания полей

%%% Математические пакеты %%%
\usepackage{amsthm,amsfonts,amsmath,amssymb,amscd}  % Математические дополнения от AMS
\usepackage{mathtools}                              % Добавляет окружение multlined

%%%% Установки для размера шрифта 14 pt %%%%
%% Формирование переменных и констант для сравнения (один раз для всех подключаемых файлов)%%
%% должно располагаться до вызова пакета fontspec или polyglossia, потому что они сбивают его работу
% \newlength{\curtextsize}
% \newlength{\bigtextsize}
% \setlength{\bigtextsize}{13.9pt}

% \makeatletter
% \show\f@size                                       % неплохо для отслеживания, но вызывает стопорение процесса, если документ компилируется без команды  -interaction=nonstopmode 
% \setlength{\curtextsize}{\f@size pt}
% \makeatother

\usepackage{polyglossia}[2014/05/21]            % Поддержка многоязычности (fontspec подгружается автоматически)

%%% Оформление абзацев %%%
\usepackage{indentfirst}                            % Красная строка

%%% Цвета %%%
% \usepackage[dvipsnames,usenames]{xcolor}
\usepackage{colortbl}
\usepackage[dvipsnames, table, hyperref, cmyk]{xcolor} % Вероятно, более новый вариант, вместо предыдущих двух строк. Конвертация всех цветов в cmyk заложена как удовлетворение возможного требования типографий. Возможно конвертирование и в rgb.
\DeclareRobustCommand{\todo}{\textcolor{red}}       % решаем проблему превращения названия цвета в результате \MakeUppercase,

%%% Таблицы %%%
\usepackage{longtable}                              % Длинные таблицы
\usepackage{multirow,makecell}                      % Улучшенное форматирование таблиц
\usepackage{tabularx}
\usepackage{multirow}

%%% Общее форматирование
\usepackage{soulutf8}                               % Поддержка переносоустойчивых подчёркиваний и зачёркиваний
\usepackage{icomma}                                 % Запятая в десятичных дробях

%%% Гиперссылки %%%
\usepackage[unicode]{hyperref}[2012/11/06]
\definecolor{linkcolor}{rgb}{0,0.6,0} % Green
% \definecolor{linkcolor}{rgb}{0,0,0} % Black
\definecolor{citecolor}{rgb}{0,0.6,0} % Green
\definecolor{urlcolor}{rgb}{0,0,1} % Blue

%%% Изображения %%%
\usepackage{graphicx}[2014/04/25]                   % Подключаем пакет работы с графикой
\graphicspath{{images/}}         % Пути к изображениям

%%% Списки %%%
\usepackage{enumitem}

% Листинги с исходным кодом программ
\usepackage{fancyvrb}
\usepackage{listings}
\usepackage{matlab-prettifier} % Подсветка синтаксиса matlab
\lccode`\~=0\relax %Без этого хака из-за особенностей пакета listings перестают работать конструкции с \MakeLowercase и т. п. в (xe|lua)latex

%%% Подписи %%%
\usepackage{caption}[2013/05/02]                    % Для управления подписями (рисунков и таблиц) % Может управлять номерами рисунков и таблиц с caption %Иногда может управлять заголовками в списках рисунков и таблиц
\usepackage{subcaption}[2013/02/03]                 % Работа с подрисунками и подобным

%%% Счётчики %%%
\usepackage[figure,table]{totalcount}               % Счётчик рисунков и таблиц
\usepackage{totcount}                               % Пакет создания счётчиков на основе последнего номера подсчитываемого элемента (может требовать дважды компилировать документ)
\usepackage{totpages}                               % Счётчик страниц, совместимый с hyperref (ссылается на номер последней страницы). Желательно ставить последним пакетом в преамбуле

\usepackage{cleveref}                           % cleveref корректно считывает язык из настроек polyglossia

\usepackage{placeins}                           % для вставки \FloatBarrier - барьера для размежения плавающих объектов

\creflabelformat{equation}{#2#1#3}                  % Формат по умолчанию ставил круглые скобки вокруг каждого номера ссылки, теперь просто номера ссылок без какого-либо дополнительного оформления

\def\slantfrac#1#2{ \hspace{3pt}\!^{#1}\!\!\hspace{1pt}/
	\hspace{2pt}\!\!_{#2}\!\hspace{3pt}
} %Макрос для красивых дробей в строчку (например, 1/2)

\lstset{ %
%    language=R,                     %  Язык указать здесь, если во всех листингах преимущественно один язык, в результате часть настроек может пойти только для этого языка
    numbers=left,                   % where to put the line-numbers
    numberstyle=\fontsize{12pt}{14pt}\selectfont\color{Gray},  % the style that is used for the line-numbers
    firstnumber=1,                  % в этой и следующей строках задаётся поведение нумерации 5, 10, 15...
    stepnumber=1,                   % the step between two line-numbers. If it's 1, each line will be numbered
    numbersep=5pt,                  % how far the line-numbers are from the code
    backgroundcolor=\color{white},  % choose the background color. You must add \usepackage{color}
    showspaces=false,               % show spaces adding particular underscores
    showstringspaces=false,         % underline spaces within strings
    showtabs=false,                 % show tabs within strings adding particular underscores
    frame=leftline,                 % adds a frame of different types around the code
    rulecolor=\color{black},        % if not set, the frame-color may be changed on line-breaks within not-black text (e.g. commens (green here))
    tabsize=2,                      % sets default tabsize to 2 spaces
    captionpos=t,                   % sets the caption-position to top
    breaklines=true,                % sets automatic line breaking
    breakatwhitespace=false,        % sets if automatic breaks should only happen at whitespace
%    title=\lstname,                 % show the filename of files included with \lstinputlisting;
    % also try caption instead of title
    basicstyle=\fontsize{12pt}{14pt}\selectfont\ttfamily,% the size of the fonts that are used for the code
%    keywordstyle=\color{blue},      % keyword style
    commentstyle=\color{ForestGreen}\emph,% comment style
    stringstyle=\color{Mahogany},   % string literal style
    escapeinside={\%*}{*)},         % if you want to add a comment within your code
    morekeywords={*,...},           % if you want to add more keywords to the set
    inputencoding=utf8,             % кодировка кода
    xleftmargin={\lmarg},           % Чтобы весь код и полоска с номерами строк была смещена влево, так чтобы цифры не вылезали за пределы текста слева
} 

\hypersetup{
    linktocpage=true,           % ссылки с номера страницы в оглавлении, списке таблиц и списке рисунков
    %    linktoc=all,                % both the section and page part are links
    %    pdfpagelabels=false,        % set PDF page labels (true|false)
    plainpages=false,           % Forces page anchors to be named by the Arabic form  of the page number, rather than the formatted form
    colorlinks,                 % ссылки отображаются раскрашенным текстом, а не раскрашенным прямоугольником, вокруг текста
    linkcolor={linkcolor},      % цвет ссылок типа ref, eqref и подобных
    citecolor={citecolor},      % цвет ссылок-цитат
    urlcolor={urlcolor},        % цвет гиперссылок
    %    hidelinks,                  % Hide links (removing color and border)
    pdflang={ru},
}

\providecommand{\e}[1]{\ensuremath{\cdot 10^{#1}}}
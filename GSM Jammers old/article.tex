\documentclass{article}
\author{{\small \textbf{Сатышев Антон Сергеевич}, старший преподаватель, satushev@gmail.com}\\
	    {\small \textbf{Ганжа Владимир Александрович}, канд. техн. наук, доцент, vladimirganzha@yandex.ru}}
\title{GSM Глушилки}
\date{}


\input{setupXETEX}
%!TEX root = article.tex
%%% Кодировки и шрифты %%%
\setmainlanguage[babelshorthands=true]{russian}  % Язык по-умолчанию русский с поддержкой приятных команд пакета babel
\setotherlanguage{english}                       % Дополнительный язык = английский (в американской вариации по-умолчанию)
\setmonofont{Courier New}
\newfontfamily\cyrillicfonttt{Courier New}
\defaultfontfeatures{Ligatures=TeX,Mapping=tex-text, Scale=1.5}
\setmainfont{Times New Roman}
\newfontfamily\cyrillicfont{Times New Roman}
\setsansfont{Arial}
\newfontfamily\cyrillicfontsf{Arial}

%%% Подписи %%%
\captionsetup{%
singlelinecheck=off,                % Многострочные подписи, например у таблиц
skip=2pt,                           % Вертикальная отбивка между подписью и содержимым рисунка или таблицы определяется ключом
justification=centering,            % Центрирование подписей, заданных командой \caption
}

%%% Рисунки %%%
%\DeclareCaptionLabelSeparator*{emdash}{~--- }             % (ГОСТ 2.105, 4.3.1)
\captionsetup[figure]{labelsep=period,position=bottom}

%%% Таблицы %%%
\DeclareCaptionFormat{tablecaption}{\centering #1#2#3} 
\captionsetup[table]{labelsep=period,position=top}                 % разделитель идентификатора с номером от наименования
\DeclareCaptionFormat{tablenocaption}{\tabcapalign #1\strut}    % Наименование таблицы отсутствует
\captionsetup[table]{singlelinecheck=off,position=top,skip=0pt}  % многострочные наименования и прочее
\DeclareCaptionLabelFormat{table}{Таблица~#2}

% Абзацный отступ. Должен быть одинаковым по всему тексту и равен пяти знакам (ГОСТ Р 7.0.11-2011, 5.3.7).
\AtBeginDocument{%
    \setlength{\parindent}{2.5em}                   
}

%%% Списки %%%
% Используем короткое тире (endash) для ненумерованных списков (ГОСТ 2.105-95, пункт 4.1.7, требует дефиса, но так лучше смотрится)
\renewcommand{\labelitemi}{\normalfont\bfseries{--}}
\setlist{nosep,%                                    % Единый стиль для всех списков (пакет enumitem), без дополнительных интервалов.
    labelindent=\parindent,leftmargin=*%            % Каждый пункт, подпункт и перечисление записывают с абзацного отступа (ГОСТ 2.105-95, 4.1.8)
}

%%% Макет страницы %%%
\geometry{a4paper,top=20mm,bottom=20mm,left=30mm,right=10mm,nofoot,nomarginpar,showframe} %
\setlength{\topskip}{0pt}   %размер дополнительного верхнего поля

%%% Колонтитулы %%%
\pagestyle{empty} % нумерация страниц выкл.

%%% Интервалы %%%
%\DoubleSpacing*     % Двойной интервал
%\OnehalfSpacing*    % Полуторный интервал
%\OneSpacing*    % Одинарный интервал
%\setSpacing{1.42}   % Полуторный интервал, подобный Ворду (возможно, стоит включать вместе с предыдущей строкой)

%%% Выравнивание и переносы %%%
%% http://tex.stackexchange.com/questions/241343/what-is-the-meaning-of-fussy-sloppy-emergencystretch-tolerance-hbadness
%% http://www.latex-community.org/forum/viewtopic.php?p=70342#p70342
\tolerance 1414
\hbadness 1414
\emergencystretch 1.5em % В случае проблем регулировать в первую очередь
\hfuzz 0.3pt
\vfuzz \hfuzz
%\raggedbottom
%\sloppy                 % Избавляемся от переполнений
\clubpenalty=10000      % Запрещаем разрыв страницы после первой строки абзаца
\widowpenalty=10000     % Запрещаем разрыв страницы после последней строки абзаца

\hypersetup{
	pdftitle={\jobname},    % Заголовок
	pdfauthor={Сатышев А.С.},  % Автор
	pdfsubject={GSM Глушилки},      % Тема
	% pdfcreator={Создатель},     % Создатель, Приложение
	%pdfproducer={Производитель},% Производитель, Производитель PDF
	pdfkeywords={}    % Ключевые слова
}

\input{biblatex}
%%% Подключение файлов bib %%%
\addbibresource[label=other]{f:/Documents AntLer/GDrive/Dissertation/Dissertation-LaTeX/biblio/othercites.bib}
\addbibresource[label=vak]{f:/Documents AntLer/GDrive/Dissertation/Dissertation-LaTeX/biblio/authorpapersVAK.bib}
\addbibresource[label=papers]{f:/Documents AntLer/GDrive/Dissertation/Dissertation-LaTeX/biblio/authorpapers.bib}
\addbibresource[label=conf]{f:/Documents AntLer/GDrive/Dissertation/Dissertation-LaTeX/biblio/authorconferences.bib}

%\usepackage[14pt]{extsizes} % для того чтобы задать нестандартный 14-ый размер шрифта

\begin{document}

	\maketitle
	\thispagestyle{empty}% нумерация титульной страницы выкл.

	\begin{abstract}
		В статье рассматриваеться возможность повышения безопасности дорожного движения, на особо опасных участках автомобильных дорог, путем орграничения мобильной связи.
		
		\textbf{Ключевые слова:} безопасность дорожного движения, базовые станции сотовой связи, транспортные средства, GSM jammer.
	\end{abstract}

	\section{Блокираторы сотовой 3G связи (джаммеры)}

	Использование блокираторов на перекрестках автодорог и опасных участках не целесообразно из-за отсутствия возможности вызвать службы спасения. Однако такие устройства довольно популярны и могут, например, применяться на АЗС, так как на них имеется персонал, следящий за безопасностью.

	\subsection{Примеры устройств}

	Аллигатор Super 100 – это стационарное устройство, питающееся от сети 220 В, 60 Вт, работающее в трех частотных диапазонах: 800, 1800 МГц и 3G с радиусом покрытия до 100 метров. Средняя рыночная стоимость 17 000 руб.
	Комплект поставки: 
	\begin{itemize}
		\item блокиратор сотовой связи Аллигатор 100;
		\item адаптер питания;
		\item три антенны;
		\item инструкция;
		\item упаковка.
	\end{itemize}
	
	Intercept 1321 "--- это джаммер предназначенный для установки на автозаправочных станциях. Эффективно подавляет сотовую сеть в радиусе до 100 метров.
	Средняя рыночная цена 80~000 руб.

	\begin{table}[ht]
	\caption{Технические характеристики}
	\label{tbl:TTX}
	\centering
		\begin{tabular}{|c|l|}
			\hline
			Производитель       & Intercept                                                              \\ \hline
			Категория           & Джаммеры                                                               \\ \hline
			Стандарт            & GSM/900/1800/850/1900 - CDMA/450/800/1900                              \\ \hline
			Частоты             & 425-475, 851/869-894, 925/935-960, 1805-1880, 1930-1990, 2110-2170 МГц \\ \hline
			На выходе           & 45 Вт                                                                  \\ \hline
			Питание             & 110/220 В                                                              \\ \hline
			Радиус действия (м) & 100                                                                    \\ \hline
			Аккумулятор         & Опционально                                                            \\ \hline
			Антенна             & Всенаправленная 3-4 dBi, Направленная до 7-8 dBi (опционально)         \\ \hline
			Кол-во каналов      & 4                                                                      \\ \hline
			Комплект поставки   & Джаммер, Антенны, блок питания                                         \\ \hline
			Размеры (ВxШxД)     & 36.5x25x48см                                                           \\ \hline
			Вес                 & 30 кг                                                                  \\ \hline
		\end{tabular}
	\end{table}

	\section{Базовые станции (БС)}

	Для ограничения возможностей сотовых телефон предлагается использовать БС, зарегистрированные у всех операторов сотовой связи в регионе установки и установленные непосредственно на перекрестках и опасных участках дороги. Такие БС не будут позволять абоненту делать вызовы и пользоваться другими сервисами сотовой сети. А так же выводить уведомляющее сообщение на дисплей мобильного телефона, что услуги связи временно недоступны. Преимущества заключается в возможности вызова служб спасения через номер 112, либо любой другой разрешенный номер.

	\subsection{Алгоритм подключения телефона к БС}

	Любой сотовый телефон все время ведет сканирование радио эфира на предмет нахождения БС с более устойчивым и мощным сигналом.

	\begin{figure}[tb]
		\centering
		\includegraphics[width=\textwidth]{BS}
		\caption{Пояснение предложенного алгоритма}
		\label{fig:BS}
	\end{figure}

	Суть предложенного метода заключается в намеренном предоставлении сотовому телефону сведений что БС установленная на опасном участке автодороги наилучшая. После подключения сотового телефона к нашей БС разговор прерывается, если в момент переключения абонент разговаривал. Также прекращается действие таких услуг как SMS, мобильный интернет и т.д. Однако, после выхода из зоны покрытия нашей БС сотовый телефон вновь подключится к БС оператора связи, а это значит все возможности коммуникации будут восстановлены.
	Минусы предложенного метода заключаются в:
	\begin{itemize}
		\item отсутствии в свободной продаже оборудования для запуска БС;
		\item необходимость в лицензировании запускаемого оборудования;
		\item высокая совокупная стоимость запуска оборудования.
	\end{itemize}
	Однако, все вышеперечисленные минусы можно с легкостью обойти, законодательно обязав операторов сотовой связи устанавливать такие БС на опасных участках автодорог.



	\FloatBarrier
	%\hypersetup{ urlcolor=black }               % Ссылки делаем чёрными
	\providecommand*{\BibDash}{}                % В стилях ugost2008 отключаем использование тире как разделителя 
	\urlstyle{rm}                               % ссылки URL обычным шрифтом
	\ifdefmacro{\microtypesetup}{\microtypesetup{protrusion=false}}{} % не рекомендуется применять пакет микротипографики к автоматически генерируемому списку литературы
	\insertbibliofull                           % Подключаем Bib-базы
	\ifdefmacro{\microtypesetup}{\microtypesetup{protrusion=true}}{}
	\urlstyle{tt}                               % возвращаем установки шрифта ссылок URL
	%\hypersetup{ urlcolor={urlcolor} }          % Восстанавливаем цвет ссылок  
\end{document}
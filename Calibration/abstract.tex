\textbf{Актуальность работы.} Для выполнения программы <<Социально-экономическое развитие Арктической зоны Российской Федерации на период до 2020 года>> утвержденной постановлением правительства \cite{PostRF} необходимо реализовать стратегию \cite{Strategi}. Согласно которой предусмотрена интеграция Арктической зоны с основными районами России посредством: освоения и разработки месторождений углеводородов, цветных и драгоценных металлов; формирования современных транспортно-логистических узлов и опорной сети автомобильных дорог; развития, реконструкции и модернизации аэропортовой сети.
% \begin{itemize}
% 	\item освоения и разработки месторождений углеводородов, цветных и драгоценных металлов;
% 	\item формирования современных транспортно-логистических узлов и опорной сети автомобильных дорог;
% 	\item развития, реконструкции и модернизации аэропортовой сети.
% \end{itemize}

Это повлечет за собой необходимость содержания, вновь построенных и реконструированных, автомобильных дорог и аэродромов в зимний период. Длительность которого в некоторых районах превышает 140 дней в году. Самые сложные и ответственные мероприятиями по содержанию дорожных покрытий, различного назначения, направлены на разрушение и удаление снежно-ледяных образований (СЛО). Известны несколько способов борьбы со СЛО: химико-механический; фрикционный; тепловой; механический. Последний способ позволяет разрушать и удалять СЛО с дорожных покрытий не нанося вреда окружающей среде, а также: экономить на химических реагентах, топливе; сохранять целостность дорожного полотна. Это закрепляет за механическим способом первенство в разработке и проектировании новых рабочих органом дорожных машин.

Однако, существует ниша в которой данный способ является мало эффективным, а именно удаление прочных снежно-ледяных образований (ПСЛО). Это обусловлено их физико-механическими свойствами: плотность $\rho = 0,6 \div 0,9 \slantfrac{\text{г}}{\text{см}^3}$; предел прочности на сжатие $\sigma = 2,5 \div 2,8$ МПа; толщина слоя $h \le 100$ мм; температура исследуемой среды $-2{}^\circ C\div-10{}^\circ C$. Существующие рабочие органы или не приспособленных для их разрушения или делают это мало эффективно. Для повышения производительности и снижения энергоемкости при удалении ПСЛО предложено применение дискового режущего инструмента \cite{GanjaDRI, GanjaPSLO, WorkOrgan}. Однако, с применением дискового режущего инструмента встает вопрос создания высокоэффективных рабочих органов, на стадии проектирования которых необходимо знать силовые параметры, величина которых зависит от множества факторов. Например, таких, как: скорость резания; геометрические параметры инструмента; температура окружающей среды и разрушаемого материала; степени износа, обусловленная величиной радиуса закругления рабочей кромки. 

\textbf{Цель работы:} подготовка инструментальной базы для выявления зависимости силовых параметров, а именно силы сопротивления резанию прочных снежно-ледяных образований, от таких факторов как радиус закругления рабочей кромки дискового режущего инструмента и шаг резания. Работа является продолжением серии экспериментальных исследований процессов взаимодействия дискового режущего инструмента с ПСЛО, в течении ряда лет, проводимых в Сибирском Федеральном Университете.

\textbf{Методы исследования:} При решении поставленной задачи применен комплексный подход, включающий: научный анализ и обобщение опыта проведенных ранее исследований; тарирование измерительного преобразователя на специальном стенде; математическую и статистическую обработку результатов тарирования.

\textbf{Результаты:} Были получены тарировочные коэффициенты преобразования напряжения с измерительного преобразователя в значение силы для каждой составляющей. Сделаны выводы о корректности подбора материала тезометрического элемента и правильности наклейки тезорезисторов. Подтверждена гипотеза о исключении взаимного влияния измеряемых составляющих друг на друга.
% для  графические зависимости переходных процессов резания льда с различным радиусом закругления рабочей кромки и шагом резания. Проанализированы полученные зависимости и сделанны выводы о корректности проведения эксперимента. А также получены данные для дальнейшего анализа и разработки методики расчета силы сопротивления резанию ПСЛО, на ранних этапах проектирования, учитывающей влияние радиуса закругления рабочей кромки, дискового режущего инструмента и шага резания.

\textbf{The urgency of the discussed issue.} To implement the government program <<Socio-economic development of the Arctic zone of the Russian Federation for the period until 2020>> approved by government decision \cite{PostRF} need to implement the strategy \cite{Strategi}. According to which the integration of the Arctic zone with the main regions of Russia is envisaged through: the exploitation of hydrocarbons, non-ferrous and precious metals; the formation of modern transport-logistics nodes and the core network of roads; development, reconstruction and modernization of the airport network.

This will erequire the maintenance, newly built and reconstructed roads and airfields during winter. The duration of which in some areas is more than 140 days a year. The most complex and important arrangements for the maintenance of road surfaces, for various purposes, aimed at the destruction of and removal of snow-ice formations (SIF). There are several ways of dealing with the SIF: chemical-mechanical; frictional; heat; mechanical. The latter method allows you to destroy and remove SIF from pavements without causing harm to the environment and to save on chemical reagents, fuel; to maintain the integrity of the roadway. It establishes mechanically the leader in the development and design of new workers on road cars.

However, there is a niche in which this method is not very effective, namely the removal of persistent snow-ice formations (PSIF). This is due to their physico-mechanical properties: density $\rho = 0,6 \div 0,9 \slantfrac{\text{g}}{\text{sm}^3}$.; the limit of compressive strength $\sigma = 2,5 \div 2,8$ MPa.; the thickness of the layer $h \le 100$ mm.; the temperature of the test medium $-2{}^\circ C\div-10{}^\circ C$. Existing working bodies or are not adapted to their destruction, or make it not effective. To improve performance and reduce energy consumption when you remove DSIF proposed the use of disk cutting tool \cite{GanjaDRI, GanjaPSLO, WorkOrgan}. However, with the use of disk cutting tool, the problem of creation of highly effective working bodies, at the design stage need to know about force paremeters, the value of which depends on many factors. Such as: cutting speed; the geometrical parameters of the instrument; ambient temperature and destructible material; value of the radius of curvature of the working edge. 

\textbf{The main aim of the study:} preparation of the instrumental base for determining the dependence of the force parameters, namely force of resistance to cutting persistent snow-ice formations on factors such as the radius of curvature of the working edge of the disk cutting tool and the step of cutting. The work is a continuation of a series of experimental researches of processes of interaction of disc cutting tools with PSIF, for a number of years, held in Siberian Federal University.

\textbf{The methods used in the study:} When solving the problem we used a comprehensive approach including: scientific analysis and generalization of the experience of earlier studies; calibration of the measuring transducer on a special test bench; mathematical and statistical processing of the results of the calibration.

\textbf{The results:} The calibration coefficients of the voltage conversion from the measuring transducer to the value of the force for each component were obtained. Conclusions are made about the correctness of the selection of the material of the tezometric element and the correctness of the paste on of the tezoresistors. The hypothesis on exclusion of mutual influence of measured components against each other is confirmed.
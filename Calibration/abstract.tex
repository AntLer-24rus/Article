\textbf{Актуальность работы.} Для выполнения программы <<Социально-экономическое развитие Арктической зоны Российской Федерации на период до 2020 года>> утвержденной постановлением правительства \cite{PostRF} необходимо реализовать стратегию \cite{Strategi}. Согласно которой предусмотрена интеграция Арктической зоны с основными районами России посредством: 
\begin{itemize}
	\item освоения и разработки месторождений углеводородов, цветных и драгоценных металлов;
	\item формирования современных транспортно-логистических узлов и опорной сети автомобильных дорог;
	\item развития, реконструкции и модернизации аэропортовой сети.
\end{itemize}

Это повлечет за собой необходимость содержания, вновь построенных и реконструированных, автомобильных дорог и аэродромов в зимний период. Длительность которого в некоторых районах превышает 140 дней в году. Самые сложные и ответственные мероприятиями по содержанию дорожных покрытий, различного назначения, направлены на разрушение и удаление снежно-ледяных образований (СЛО). Известны несколько способов борьбы со СЛО: химико-механический; фрикционный; тепловой; механический. Последний способ позволяет разрушать и удалять СЛО с дорожных покрытий не нанося вреда окружающей среде, а также: экономить на химических реагентах, топливе; сохранять целостность дорожного полотна. Это закрепляет за механическим способом первенство в разработке и проектировании новых рабочих органом дорожных машин.

Однако, существует ниша в которой данный способ является мало эффективным, а именно удаление прочных снежно-ледяных образований (ПСЛО). Это обусловлено тем что их прочность значительно выше \todo{Цыфры} и существующие рабочие органы или не приспособленных для их разрушения или делают это мало эффективно. Для повышения эффективности и снижения энергоемкости при удалении ПСЛО предложено применение дискового режущего инструмента \cite{GanjaDRI, GanjaPSLO, WorkOrgan}. Однако, с применением дискового режущего инструмента встает вопрос создания высокоэффективных рабочих органов, на стадии проектирования которых необходимо знать силовые параметры, величина которых зависит от множества факторов. Например, таких, как: скорость резания; геометрические параметры инструмента; температура окружающей среды и разрушаемого материала; степени износа, обусловленная радиусом закругления рабочей кромки. 

\textbf{Цель работы:} преследует выявление зависимости силовых параметров, а именно силы сопротивления резанию прочных снежно-ледяных образований, от таких факторов как радиус закругления рабочей кромки дискового режущего инструмента и шаг резания. Работа является продолжением серии экспериментальных исследований проводимых в Сибирском Федеральном Университете.

\textbf{Методы исследования:} При решении поставленной задачи применен комплексный подход, включающий: научный анализ и обобщение опыта проведенных ранее исследований; экспериментальные лабораторные исследования процесса резания льда полноразмерными дисковыми режущими инструментами с различным радиусом закругления рабочей кромки; математическую и статистическую обработку результатов эксперимента.

\textbf{Результаты:} Были получены графические зависимости переходных процессов резания льда с различным радиусом закругления рабочей кромки и шагом резания. Проанализированы полученные зависимости и сделанны выводы о корректности проведения эксперимента. А также получены данные для дальнейшего анализа и построения математической модели процесса резания льда учитывающей влияние радиуса закругления рабочей кромки, дискового режущего инструмента и шага резания.
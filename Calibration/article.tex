\documentclass[a4paper,12pt]{article}

%!TEX root = article.tex
\usepackage[noblocks]{authblk}                      % Для добавления ссылок на авторов
% после знаков препинания пробелы не увеличиваются
\frenchspacing

%%% Рисунки в текста %%%
\usepackage{wrapfig}

\RequirePackage{etoolbox}[2015/08/02]               % Для продвинутой проверки разных условий

%%% Поля и разметка страницы %%%
\usepackage{pdflscape}                              % Для включения альбомных страниц
\usepackage{geometry}                               % Для последующего задания полей

%%% Математические пакеты %%%
\usepackage{amsthm,amsfonts,amsmath,amssymb,amscd}  % Математические дополнения от AMS
\usepackage{mathtools}                              % Добавляет окружение multlined

%%%% Установки для размера шрифта 14 pt %%%%
%% Формирование переменных и констант для сравнения (один раз для всех подключаемых файлов)%%
%% должно располагаться до вызова пакета fontspec или polyglossia, потому что они сбивают его работу
% \newlength{\curtextsize}
% \newlength{\bigtextsize}
% \setlength{\bigtextsize}{13.9pt}

% \makeatletter
% \show\f@size                                       % неплохо для отслеживания, но вызывает стопорение процесса, если документ компилируется без команды  -interaction=nonstopmode 
% \setlength{\curtextsize}{\f@size pt}
% \makeatother

\usepackage{polyglossia}[2014/05/21]            % Поддержка многоязычности (fontspec подгружается автоматически)

%%% Оформление абзацев %%%
\usepackage{indentfirst}                            % Красная строка

%%% Цвета %%%
% \usepackage[dvipsnames,usenames]{xcolor}
\usepackage{colortbl}
\usepackage[dvipsnames, table, hyperref, cmyk]{xcolor} % Вероятно, более новый вариант, вместо предыдущих двух строк. Конвертация всех цветов в cmyk заложена как удовлетворение возможного требования типографий. Возможно конвертирование и в rgb.
\DeclareRobustCommand{\todo}{\textcolor{red}}       % решаем проблему превращения названия цвета в результате \MakeUppercase,

%%% Таблицы %%%
\usepackage{longtable}                              % Длинные таблицы
\usepackage{multirow,makecell}                      % Улучшенное форматирование таблиц
\usepackage{tabularx}
\usepackage{multirow}

%%% Общее форматирование
\usepackage{soulutf8}                               % Поддержка переносоустойчивых подчёркиваний и зачёркиваний
\usepackage{icomma}                                 % Запятая в десятичных дробях

%%% Гиперссылки %%%
\usepackage[unicode]{hyperref}[2012/11/06]
\definecolor{linkcolor}{rgb}{0,0.6,0} % Green
% \definecolor{linkcolor}{rgb}{0,0,0} % Black
\definecolor{citecolor}{rgb}{0,0.6,0} % Green
\definecolor{urlcolor}{rgb}{0,0,1} % Blue

%%% Изображения %%%
\usepackage{graphicx}[2014/04/25]                   % Подключаем пакет работы с графикой
\graphicspath{{images/}}         % Пути к изображениям

%%% Списки %%%
\usepackage{enumitem}

% Листинги с исходным кодом программ
\usepackage{fancyvrb}
\usepackage{listings}
\usepackage{matlab-prettifier} % Подсветка синтаксиса matlab
\lccode`\~=0\relax %Без этого хака из-за особенностей пакета listings перестают работать конструкции с \MakeLowercase и т. п. в (xe|lua)latex

%%% Подписи %%%
\usepackage{caption}[2013/05/02]                    % Для управления подписями (рисунков и таблиц) % Может управлять номерами рисунков и таблиц с caption %Иногда может управлять заголовками в списках рисунков и таблиц
\usepackage{subcaption}[2013/02/03]                 % Работа с подрисунками и подобным

%%% Счётчики %%%
\usepackage[figure,table]{totalcount}               % Счётчик рисунков и таблиц
\usepackage{totcount}                               % Пакет создания счётчиков на основе последнего номера подсчитываемого элемента (может требовать дважды компилировать документ)
\usepackage{totpages}                               % Счётчик страниц, совместимый с hyperref (ссылается на номер последней страницы). Желательно ставить последним пакетом в преамбуле

\usepackage{cleveref}                           % cleveref корректно считывает язык из настроек polyglossia

\usepackage{placeins}                           % для вставки \FloatBarrier - барьера для размежения плавающих объектов

\creflabelformat{equation}{#2#1#3}                  % Формат по умолчанию ставил круглые скобки вокруг каждого номера ссылки, теперь просто номера ссылок без какого-либо дополнительного оформления

\def\slantfrac#1#2{ \hspace{3pt}\!^{#1}\!\!\hspace{1pt}/
	\hspace{2pt}\!\!_{#2}\!\hspace{3pt}
} %Макрос для красивых дробей в строчку (например, 1/2)

\lstset{ %
%    language=R,                     %  Язык указать здесь, если во всех листингах преимущественно один язык, в результате часть настроек может пойти только для этого языка
    numbers=left,                   % where to put the line-numbers
    numberstyle=\fontsize{12pt}{14pt}\selectfont\color{Gray},  % the style that is used for the line-numbers
    firstnumber=1,                  % в этой и следующей строках задаётся поведение нумерации 5, 10, 15...
    stepnumber=1,                   % the step between two line-numbers. If it's 1, each line will be numbered
    numbersep=5pt,                  % how far the line-numbers are from the code
    backgroundcolor=\color{white},  % choose the background color. You must add \usepackage{color}
    showspaces=false,               % show spaces adding particular underscores
    showstringspaces=false,         % underline spaces within strings
    showtabs=false,                 % show tabs within strings adding particular underscores
    frame=leftline,                 % adds a frame of different types around the code
    rulecolor=\color{black},        % if not set, the frame-color may be changed on line-breaks within not-black text (e.g. commens (green here))
    tabsize=2,                      % sets default tabsize to 2 spaces
    captionpos=t,                   % sets the caption-position to top
    breaklines=true,                % sets automatic line breaking
    breakatwhitespace=false,        % sets if automatic breaks should only happen at whitespace
%    title=\lstname,                 % show the filename of files included with \lstinputlisting;
    % also try caption instead of title
    basicstyle=\fontsize{12pt}{14pt}\selectfont\ttfamily,% the size of the fonts that are used for the code
%    keywordstyle=\color{blue},      % keyword style
    commentstyle=\color{ForestGreen}\emph,% comment style
    stringstyle=\color{Mahogany},   % string literal style
    escapeinside={\%*}{*)},         % if you want to add a comment within your code
    morekeywords={*,...},           % if you want to add more keywords to the set
    inputencoding=utf8,             % кодировка кода
    xleftmargin={\lmarg},           % Чтобы весь код и полоска с номерами строк была смещена влево, так чтобы цифры не вылезали за пределы текста слева
} 

\hypersetup{
    linktocpage=true,           % ссылки с номера страницы в оглавлении, списке таблиц и списке рисунков
    %    linktoc=all,                % both the section and page part are links
    %    pdfpagelabels=false,        % set PDF page labels (true|false)
    plainpages=false,           % Forces page anchors to be named by the Arabic form  of the page number, rather than the formatted form
    colorlinks,                 % ссылки отображаются раскрашенным текстом, а не раскрашенным прямоугольником, вокруг текста
    linkcolor={linkcolor},      % цвет ссылок типа ref, eqref и подобных
    citecolor={citecolor},      % цвет ссылок-цитат
    urlcolor={urlcolor},        % цвет гиперссылок
    %    hidelinks,                  % Hide links (removing color and border)
    pdflang={ru},
}

\providecommand{\e}[1]{\ensuremath{\cdot 10^{#1}}}
\renewcommand\Authfont{\small}
\renewcommand\Authsep{, \\}
\renewcommand\Authand{, \\}
\renewcommand\Authands{, \\}

%%% Кодировки и шрифты %%%
\setmainlanguage[babelshorthands=true]{russian}  % Язык по-умолчанию русский с поддержкой приятных команд пакета babel
\setotherlanguage{english}                       % Дополнительный язык = английский (в американской вариации по-умолчанию)
\setmonofont{Courier New}
\newfontfamily\cyrillicfonttt{Courier New}
\defaultfontfeatures{Ligatures=TeX,Mapping=tex-text}
\setmainfont{Times New Roman}
\newfontfamily\cyrillicfont{Times New Roman}
\setsansfont{Arial}
\newfontfamily\cyrillicfontsf{Arial}

%%% Подписи %%%
\captionsetup{%
singlelinecheck=off,                % Многострочные подписи, например у таблиц
skip=2pt,                           % Вертикальная отбивка между подписью и содержимым рисунка или таблицы определяется ключом
justification=centering,            % Центрирование подписей, заданных командой \caption
}

%%% Рисунки %%%
%\DeclareCaptionLabelSeparator*{emdash}{~--- }             % (ГОСТ 2.105, 4.3.1)
\captionsetup[figure]{labelsep=period,position=bottom}

%%% Таблицы %%%
%\DeclareCaptionFormat{tablecaption}{\centering #1#2#3} 
\captionsetup[table]{labelsep=period,position=top}                 % разделитель идентификатора с номером от наименования
%\DeclareCaptionFormat{tablenocaption}{\tabcapalign #1\strut}    % Наименование таблицы отсутствует
\captionsetup[table]{singlelinecheck=off,position=top,skip=0pt}  % многострочные наименования и прочее
%\DeclareCaptionLabelFormat{table}{Таблица~#2}


\AtBeginDocument{%
    \setlength{\parindent}{2.5em}                   % Абзацный отступ. Должен быть одинаковым по всему тексту и равен пяти знакам (ГОСТ Р 7.0.11-2011, 5.3.7).
}

%%% Списки %%%
% Используем короткое тире (endash) для ненумерованных списков (ГОСТ 2.105-95, пункт 4.1.7, требует дефиса, но так лучше смотрится)
\renewcommand{\labelitemi}{\normalfont\bfseries{--}}
\setlist{nosep,%                                    % Единый стиль для всех списков (пакет enumitem), без дополнительных интервалов.
    labelindent=\parindent,leftmargin=*%            % Каждый пункт, подпункт и перечисление записывают с абзацного отступа (ГОСТ 2.105-95, 4.1.8)
}

%%% Заголовки %%%
\makeatletter
\renewcommand{\section}{\@startsection{section}{1}%
{0pt}%отступ заголовка от левого поля
{1ex}% величина вертикального отступа, оставляемого перед заголовком.
{1ex}% величиа вертикального отступа после заголовка
{\normalfont\bfseries}}%стиль оформления заголовка
\makeatother
\makeatletter
\renewcommand{\subsection}{\@startsection{subsection}{1}%
{0pt}%отступ заголовка от левого поля
{1ex}% величина вертикального отступа, оставляемого перед заголовком.
{1ex}% величиа вертикального отступа после заголовка
{\normalfont\bfseries}}%стиль оформления заголовка
\makeatother

\AtBeginDocument{%
   \def\postsection{~}% Один пробел без точки после номера секции
   \def\postsubsection{~}
}

%%% Авторы и места %%%
\renewcommand\Affilfont{\normalsize}
\renewcommand\Authfont{\normalsize}
%%% Заглавие статьи %%%
\makeatletter
    \def\@maketitle{%
  \newpage%
  \begin{center}%
    {\normalfont\bfseries\MakeUppercase{\@title}}
  \end{center}%
  \@author\par
  }
\makeatother

%%% Формат блока аннотации %%%
\renewenvironment{abstract}{%
	\begin{center}
  		\bfseries\abstractname\vspace{-.5em}\vspace{0pt}
  	\end{center}
	  \list{}{%
	    \setlength{\listparindent}{\parindent} % Абзацный отступ аннотации
	  	\setlength{\itemindent }{\parindent}
	    \setlength{\leftmargin}{0mm} % Поля аннотации
	    \setlength{\rightmargin}{\leftmargin}%
	    \setlength{\parsep}{\parskip}% Отступ ежду абзацами
	  }%
	  \item\relax}
{\endlist}

%%% Макет страницы %%%
\geometry{a4paper,top=25mm,bottom=25mm,left=25mm,right=25mm,nofoot,nomarginpar} %,showframe
\setlength{\topskip}{0pt}   %размер дополнительного верхнего поля

%%% Колонтитулы %%%
\pagestyle{empty} % нумерация страниц выкл.

%%% Интервалы %%%
%\DoubleSpacing*     % Двойной интервал
% \OnehalfSpacing*    % Полуторный интервал
%\OneSpacing*    % Одинарный интервал
%\setSpacing{1.42}   % Полуторный интервал, подобный Ворду (возможно, стоит включать вместе с предыдущей строкой)

%%% Выравнивание и переносы %%%
%% http://tex.stackexchange.com/questions/241343/what-is-the-meaning-of-fussy-sloppy-emergencystretch-tolerance-hbadness
%% http://www.latex-community.org/forum/viewtopic.php?p=70342#p70342
\tolerance 1414
\hbadness 1414
\emergencystretch 1.5em % В случае проблем регулировать в первую очередь
\hfuzz 0.3pt
\vfuzz \hfuzz
%\raggedbottom
% \sloppy                 % Избавляемся от переполнений
\clubpenalty=10000      % Запрещаем разрыв страницы после первой строки абзаца
\widowpenalty=10000     % Запрещаем разрыв страницы после последней строки абзаца

\newcommand{\mainAuthor}{Сатышев Антон Сергеевич}
\newcommand{\articleTitle}{Результаты эксперимента}
\newcommand{\keywords}{снежно-ледяные образования, дисковый режущий инструмент, силовые параметры, радиус закругления, лёд, радиус закругления рабой кромки}


\author{\mainAuthor, старший преподаватель, satushev@gmail.com}
\author{Ганжа Владимир Александрович, канд. техн. наук, доцент, vladimirganzha@yandex.ru}
\affil{Институт нефти а газа <<Сибирский Федеральный Университет>>, 660041, г. Красноярск, Свободный проспект, 82 ст. 6}

\title{\articleTitle}
\date{}

\hypersetup{
	pdftitle={\articleTitle},    % Заголовок
	pdfauthor={\mainAuthor},  % Автор
	% pdfsubject={Результаты эксперимента},      % Тема
	% pdfcreator={Создатель},     % Создатель, Приложение
	% pdfproducer={Производитель},% Производитель, Производитель PDF
	pdfkeywords={\keywords}    % Ключевые слова
}

\input{../Setups/biblatex}


%%% Подключение файлов bib %%%
\addbibresource[label=other]{f:/Documents AntLer/GDrive/Dissertation/Dissertation-LaTeX/biblio/othercites.bib}
\addbibresource[label=vak]{f:/Documents AntLer/GDrive/Dissertation/Dissertation-LaTeX/biblio/authorpapersVAK.bib}
\addbibresource[label=papers]{f:/Documents AntLer/GDrive/Dissertation/Dissertation-LaTeX/biblio/authorpapers.bib}
\addbibresource[label=conf]{f:/Documents AntLer/GDrive/Dissertation/Dissertation-LaTeX/biblio/authorconferences.bib}

%\geometry{showframe} % Отображать поля страницы

\begin{document}
	\renewcommand{\tablename}{Таблица.}
	\hypersetup{urlcolor=black}               % Ссылки делаем чёрными
	\maketitle % Создание заголовка статьи
	\thispagestyle{empty}% нумерация титульной страницы выкл.
	\begin{abstract}
		% Аннотация находится в файле abstract.tex
		\textbf{Актуальность работы.} Для выполнения программы <<Социально-экономическое развитие Арктической зоны Российской Федерации на период до 2020 года>> утвержденной постановлением правительства \cite{PostRF} необходимо реализовать стратегию \cite{Strategi}. Согласно которой предусмотрена интеграция Арктической зоны с основными районами России посредством: 
\begin{itemize}
	\item освоения и разработки месторождений углеводородов, цветных и драгоценных металлов;
	\item формирования современных транспортно-логистических узлов и опорной сети автомобильных дорог;
	\item развития, реконструкции и модернизации аэропортовой сети.
\end{itemize}

Это повлечет за собой необходимость содержания, вновь построенных и реконструированных, автомобильных дорог и аэродромов в зимний период. Длительность которого в некоторых районах превышает 140 дней в году. Самые сложные и ответственные мероприятиями по содержанию дорожных покрытий, различного назначения, направлены на разрушение и удаление снежно-ледяных образований (СЛО). Известны несколько способов борьбы со СЛО: химико-механический; фрикционный; тепловой; механический. Последний способ позволяет разрушать и удалять СЛО с дорожных покрытий не нанося вреда окружающей среде, а также: экономить на химических реагентах, топливе; сохранять целостность дорожного полотна. Это закрепляет за механическим способом первенство в разработке и проектировании новых рабочих органом дорожных машин.

Однако, существует ниша в которой данный способ является мало эффективным, а именно удаление прочных снежно-ледяных образований (ПСЛО). Это обусловлено тем что их прочность значительно выше \todo{Цыфры} и существующие рабочие органы или не приспособленных для их разрушения или делают это мало эффективно. Для повышения эффективности и снижения энергоемкости при удалении ПСЛО предложено применение дискового режущего инструмента \cite{GanjaDRI, GanjaPSLO, WorkOrgan}. Однако, с применением дискового режущего инструмента встает вопрос создания высокоэффективных рабочих органов, на стадии проектирования которых необходимо знать силовые параметры, величина которых зависит от множества факторов. Например, таких, как: скорость резания; геометрические параметры инструмента; температура окружающей среды и разрушаемого материала; степени износа, обусловленная радиусом закругления рабочей кромки. 

\textbf{Цель работы:} преследует выявление зависимости силовых параметров, а именно силы сопротивления резанию прочных снежно-ледяных образований, от таких факторов как радиус закругления рабочей кромки дискового режущего инструмента и шаг резания. Работа является продолжением серии экспериментальных исследований проводимых в Сибирском Федеральном Университете.

\textbf{Методы исследования:} При решении поставленной задачи применен комплексный подход, включающий: научный анализ и обобщение опыта проведенных ранее исследований; экспериментальные лабораторные исследования процесса резания льда полноразмерными дисковыми режущими инструментами с различным радиусом закругления рабочей кромки; математическую и статистическую обработку результатов эксперимента.

\textbf{Результаты:} Были получены графические зависимости переходных процессов резания льда с различным радиусом закругления рабочей кромки и шагом резания. Проанализированы полученные зависимости и сделанны выводы о корректности проведения эксперимента. А также получены данные для дальнейшего анализа и построения математической модели процесса резания льда учитывающей влияние радиуса закругления рабочей кромки, дискового режущего инструмента и шага резания.

		\textbf{Ключевые слова:} \keywords.		
	\end{abstract}
	% Основная часть статьи находится в файле content.tex
	\section{Введение}

Для более объективного изучения процесса взаимодействия дискового инструмента с ПСЛО предлагается контролировать три составляющие силы резания: горизонтальную, боковую и вертикальную. Контроль этих составляющих непосредственно на рабочем органе мало эффективен, так как: требует больших трудозатрат и дорогостоящего оборудования (датчики силы, оснастка для их монтажа); невозможно изолировать влияние температуры окружающей среды, влажности, теплозапаса дорожного полотна и других факторов друг на друга; постоянно меняются физико-механические свойства ПСЛО (прочность, плотность, наличие абразивного материала). Поэтому, опираясь на результаты работ по резанию мерзлых грунтов различными инструментами \cite{JelukevichGrunt, BaronTang, BaronShar, Zelenin}, целесообразно исследовать процесс взаимодействия полноразмерного дискового режущего инструмента с различным радиусом закругления рабочей кромки с разрушаемым массивом путем стендовых испытаний в лабораторных условиях.

В качестве режущего инструмента принят заостренный дисковый	резец изображенный на рисунке \ref{fig:DRI}.
\begin{figure}[ht]
	\centering
	\includegraphics[width=0.5\textwidth]{DRI_Final}

	$t$ "--- шаг резания; $D$ "--- диаметр дискового резца; $\delta$ "--- угол заострения; $h$ "--- глубина резания; $\gamma$ "--- задний угол. 
	\caption{Схема взаимодействия дискового режущего инструмента с разрушаемым массивом} 
	\label{fig:DRI}
	Fig. \ref{fig:DRI}. Scheme of interaction of a disk cutting tool with a destructible array
\end{figure}
При проведении экспериментальных исследований использовались дисковые резцы с различным радиусом закругления рабочей кромки. $ R=[0,5; 1,5; 2,5; 3,5; 4,5] $ мм. Данный диапазон значений обусловлен результатами исследованиями изнашивания режущей кромки проведенными в работе \cite{BaronTang}. Остальные параметры дискового режущего инструмента приняты следующими: диаметр:  $D=200$ мм.; угол заострения: $\delta=30^\circ$; глубина резания: $h=60$ мм.; шаг резания: $t=[10; 20; 30; 40; 50]$ мм.; задний угол: $\gamma=3^\circ\div5^\circ$; температура окружающего воздуха: $-2{}^\circ C\div-7{}^\circ C$; скорость резания: $0,51\ \slantfrac{\text{м}}{\text{c}}$ ($1,84\ \slantfrac{\text{км}}{\text{ч}}$).

Для проведения эксперимента использовался механизированный лабораторный стенд описанный в работе \cite{Sram2013Modernizaciya} и защищенный патентом на изобретение №~2429459 \cite{ExpStend}. Для фиксирования, сбора и записи информации применен измерительный комплекс описанный с статье \cite{IKI2016:my}.

% \section{Тарирование тензометрического звена}

% Для тарирования тензометрического звена \cite{IKI2016:my} применялся стенд \cite{CalibrationStend}, позволяющий закреплять звено в различных положения и соответственно создать требуемый вектор нагрузки. Тарирование производилось с помощью: одного измерительного прибора "--- динамометра растяжения ДПУ-5-2~5033 второго класса точности; талрепа и вспомогательной оснастки для крепежа тензометрического звена.

% Нагрузка звена осуществлялась ступенчато, с шагом 500~Н, до предельного значения в 2~500~Н. Разгрузка производилась с тем же шагом до нулевого значения. На рисунке~\ref{fig:CalibrationRawHor} приведены графики переходных процессов возникающих во время тарирования. Из графиков явно видно что исключено взаимное влияние составляющих друг на друга.

% \begin{figure} [ht]
% 	\centering
% 	\includegraphics[width=0.7\textwidth]{CalibrationRawHor}
	
% 	С веху вниз: горизонтальная, боковая, вертикальная составляющие усилия резания
% 	\caption{Графики переходных процессов при тарировании горизонтальной составляющей усилия резания}  
% 	\label{fig:CalibrationRawHor}  
% \end{figure}

% Используя данные с графиков переходных процессов можно получить зависимости изображенные на рисунке~\ref{fig:Calibration}. Из них видно что силы возникающие на тензометрическом звене имеют линейную зависимость от напряжения получаемого с тензометрических мостов. 

% \begin{figure} [ht]
% 	\centering
% 	\includegraphics[width=0.7\textwidth]{Calibration}
	
% 	1,2,3 "--- горизонтальная, бокова, вертикальная составляющие усилия резания соответственно при тарировании горизонтальной составляющей; 4,5,6 "--- аналогично  при тарировании боковой составляющей; 7,8,9 "--- аналогично при тарировании вертикальной составляющей.
% 	\caption{Графики тарирования тензометрического звена}
% 	\label{fig:Calibration}  
% \end{figure}

% Эти графики можно представить в виде уравнений:
% \begin{align}
% 	y_\text{гор}  & = 80074.568 \cdot x  \label{eq:TrendHor}\\
% 	y_\text{бок}  & = 140953.396 \cdot x \label{eq:TrendLat}\\
% 	y_\text{верт} & = 51338.284 \cdot x \label{eq:TrendVert}
% \end{align}
% Из уравнений \labelcref{eq:TrendHor,eq:TrendLat,eq:TrendVert} получим тарировочные коэффициенты: 8~0074.568~$ \slantfrac{\text{Н}}{\text{В}} $, 140~953.396~$ \slantfrac{\text{Н}}{\text{В}} $, 51~338.284~$ \slantfrac{\text{Н}}{\text{В}} $ для горизонтальной, боковой и вертикальной составляющей усилия резания соответственно.

\section{Обработка результатов эксперимента}

После проведения экспериментальных лабораторных исследований выявления влияния радиуса закругления рабочей кромки и шага резания на составляющие силы, возникающей на дисковом инструменте, при механическом разрушении льда, получен набор файлов с записью значений напряжений снятых с АЦП. Каждый файл соответствует своему сочетанию исследуемых параметров \textit{R} и \textit{t}. структура файла приведена на рисунке \ref{fig:FileMap}.
\begin{figure}[ht]
	\centering
	\includegraphics[scale=0.999]{DataFormat}%,width=1\textwidth

	1 "--- Название блока данных; 2 "--- Дата и время начала измерений; 3 "--- Интервал между измерениями в миллисекундах; 4 "--- Количество сохранённых измерений; 5 "--- Отсчёт времени в секундах; 6, 7, 8 "--- Значение напряжения на чувствительном элементе в вольтах для первого, второго и третьего канала соответственно.
	\caption{Пример структуры файла хранения данных}
	\label{fig:FileMap}
	Fig. \ref{fig:FileMap}. Example of the structure of a data storage file
\end{figure}
Для дальнейшего использования, полученных данных, предлагаеться произвести обработку и оценку их корректности методами математики и статистики, такими как:отброс грубых ошибок; фильтрация; сглаживание; отброс постоянной составляющей; усреднение значений повторных экспериментов.

% \begin{itemize}
% 	\item отброс грубых ошибок;
% 	\item фильтрация;
% 	\item сглаживание;
% 	\item отброс постоянной составляющей;
% 	\item усреднение значений повторных экспериментов.
% \end{itemize}

\subsection{Алгоритм отброса грубых ошибок}

Для улучшения точности оценки переходного процесса и снижения влияния всевозможных внешних факторов целесообразно применить к полученному набору точек (сигналу) алгоритм отброса грубых ошибок \cite{LvovStat}.
Суть алгоритма заключается в использовании метода максимального относительного отклонения:
\begin{equation}\label{eq:MOO}
	\tau=\frac{|x_i-\bar{x}|}{\sigma_x},
\end{equation} 
где $ x_i $ "--- крайний (наибольший или наименьший) элемент сигнала; $ \bar{x} $ "--- среднее значение сигнала; $ \sigma_x $ "--- СКО сигнал.

Сравнивая $ \tau $ с критическим значением $ \tau_{(p,n)} $, рассчитанным по формуле~\ref{eq:tau_krit}, можно сделать вывод является ли наблюдение грубой погрешностью или нет.
\begin{equation}\label{eq:tau_krit}
	\tau_{(p,\ n)}=\frac{t_{(p,\ n-2)}\cdot\sqrt{n-1}}{\sqrt{n-2+|t_{(p,\ n-2)}|^2}}
\end{equation}
где $ t_{(p,\ n-2)} $ "--- критическое значение распределения Стьюдента при доверительной вероятность $ q=1-p $; $ n $ "--- количество наблюдений в сигнале переходного процесса.

% Критическое значение распределения Стьюдента в ППП matlab можно вычислить с помощью функции \lstinline{tinv()}. 
% \begin{lstlisting}
% t = tinv(p / 100, n - 2);
% \end{lstlisting}
% где \lstinline{p} "--- вероятность; \lstinline{n} "--- количество наблюдений.

Таким образом имеем алгоритм отброса грубых ошибок представленный на рисунке~\ref{fig:AlgDGE}.
\begin{figure}[!htb]
	\centering
	\includegraphics[scale=1.25]{images/DropGrossError}
	\caption{Алгоритм отброса грубых ошибок} 
	\label{fig:AlgDGE}
	Fig. \ref{fig:AlgDGE}. Algorithm for drop gross errors
\end{figure}
Имеет смысл ввести три группы наблюдений, удовлетворяющих следующим условиям: $ \tau \leqslant \tau_{(5\%,\ n)} $ нельзя отсеивать; $ \tau_{(5\%,\ n)} < \tau < \tau_{(0,1\%,\ n)} $ можно отсеять, если в пользу этой процедуры имеются и другие соображения; $ \tau > \tau_{(0,1\%,\ n)} $ отсеиваются всегда.
% \begin{itemize}
% 	\item $ \tau \leqslant \tau_{(5\%,\ n)} $ нельзя отсеивать.
% 	\item $ \tau_{(5\%,\ n)} < \tau < \tau_{(0,1\%,\ n)} $ можно отсеять, если в пользу этой процедуры имеются и другие соображения.
% 	\item $ \tau > \tau_{(0,1\%,\ n)} $ отсеиваются всегда.
% \end{itemize}
Приведем объяснение работы каждого блока в вербальном виде:
\begin{enumerate}
	\item Из наблюдаемых значений выбирается максимальное и минимальное значение сигнала по модулю, далее значение сравниваются и выбирается наибольшее.\label{enum:alg_DGE_1}
	\item Рассчитывается $ \tau $ по формуле~\ref{eq:MOO}.
	\item Вычисляются критические точки $ \tau_{(0,1\%,\ n)} $ и $ \tau_{(5\%,\ n)} $ по формуле~\ref{eq:tau_krit}.
	\item Проверяется условие $ \tau > \tau_{(0,1\%,\ n)} $ если выполняется переходим к пункту~\ref{enum:alg_DGE_5}, иначе к~\ref{enum:alg_DGE_6}.
	\item Исключаем наблюдение из массива точек и переходим к пункту~\ref{enum:alg_DGE_1}.\label{enum:alg_DGE_5}
	\item Проверяем условие $ \tau \leqslant \tau_{(5\%,\ n)} $ если выполняется переходим к пункту~\ref{enum:alg_DGE_9}, иначе к~\ref{enum:alg_DGE_7}.\label{enum:alg_DGE_6}
	\item Анализируем другие факторы способные указать на допущение грубой ошибки.\label{enum:alg_DGE_7}
	\item Принимаем решение отбрасывать или нет. Если отбрасываем переходим к пункту~\ref{enum:alg_DGE_5}, иначе к~\ref{enum:alg_DGE_9}.\label{enum:alg_DGE_8}
	\item Выход из алгоритма. Оставшиеся наблюдения и есть полезный сигнал.\label{enum:alg_DGE_9}
\end{enumerate}

\subsection{Сглаживание сигнала}

Для более наглядной читаемости и устранения влияния высокочастотных помех, предлагается применение цифрового фильтра <<Скользящая средняя>>. Этот способ является наиболее простым в реализации и даёт хорошие результаты при правильном подборе апертуры. Скользящее среднее (moving average \textbf{MA}) вычисляется по формуле:

\begin{equation}
	% \begin{multlined}
	\begin{aligned}
		MA_t &= b_0 \cdot p_t+b_1 \cdot p_{t-1}+\cdots+b_i \cdot p_{t-i}+\cdots+b_{n-2} \cdot p_{t-n-2}+b_{n-1} \cdot p_{t-n-1} = \\
		&= \frac{1}{n}\sum_{i=0}^{n-1}\left( b_i \cdot p_{t-i}\right),
	\end{aligned}
	% \end{multlined}
\end{equation}
где $ MA_t $ "--- значение скользящего среднего в точке $ t $; $ n $ "--- количество значений исходной функции для расчёта скользящего среднего (апертура); $ p_{t-i} $ "--- значение исходной функции в точке $ t-i $; $ b_i $ "--- вектор весовых коэффициентов.

Обычно для фильтров скользящего среднего применяется равномерное распределение весов. Например, если $ n = 4 $, то $ b = \left[ \frac{1}{4}\ \frac{1}{4}\ \frac{1}{4}\ \frac{1}{4}\right] $. Такой фильтр будет называться простым скользящим средним (simple moving average SMA).
В данной работе предлагается выбирать весовые коэффициенты для MA путём оценки автокорреляционной функции от требуемого сигнала. На рисунке~\ref{fig:Autocorr} построена автокорреляционная функция и ее доверительные интервалы $ [-0.09623\ 0.09623] $.
\begin{figure}[!ht] 
	\centering
	\includegraphics[scale=0.835]{Autocorr}%width=1\textwidth
	\caption{Автокорреляционная функция с доверительным интервалом} 
	\label{fig:Autocorr} 
	Fig. \ref{fig:Autocorr}. Autocorrelation function with confidence interval
\end{figure}
Предлагается за весовые коэффициенты взять первые значения автокорреляционной функции, до пересечения её и <<верхней>> доверительной границы.
Такой алгоритм обусловлен включением в окно скользящего среднего только <<тесно>> связанных между собой значений целевой функции, и в тоже время применяет веса к включённым значениям согласно их влиянию.

\section{Выводы}

Анализируя работу описанных алгоритмов, по рисунку \ref{fig:SignalRaw} можно видеть отброшенные пиковые выбросы сигнала 2, рассчитанные с помощью алгоритма отброса грубых ошибок.
\begin{figure}[!ht] 
	\centering
	\includegraphics[scale=0.85]{SignalRaw}%width=1\textwidth
	
	1 "--- сигнал полученный с АЦП <<сырой>>; 2 "--- отброшенные точки в результате работы алгоритма отброса грубых ошибок; 3 "--- результат скользящего среднего; 4 "--- результат отброса постоянной составляющей.
	\caption{Переходный процесса разрушения льда} 
	\label{fig:SignalRaw} 
	Fig. \ref{fig:SignalRaw}. Transient process of ice breakdown
\end{figure}
Такие выбросы могут свидетельствовать о кратковременных скачках напряжения в питающей сети, обусловленных работой силового оборудования, такого как трехфазный двигатель привода лабораторного стенда, холодильная установка.

Так же на графике представлен сглаженный сигнал переходного процесса резания льда 3, полученный путем применения алгоритма скользящего среднего с адаптивным окном сглаживания. Как видно график 3 имеет некоторое смещение по временной оси, которое обусловлено размером окна сглаживания. Смещение не является критичны, так как расположено в начале временной оси, в тот момент времени когда, происходит движение резца в свободной состоянии (до момента внедрения в ледяной массив). Так же на графике 3 явно видны отрицательные значения сигнала. Отрицательные значения на временном промежутке от 0,375 до 1,75 секунд объясняются наличием упругих элементов в тензометрическом звене и не нулевыми моментами инерции кронштейна и дискового режущего инструмента. Отрицательные значения на промежутке времени с 2,75 по 3,5 секунд имеют тот же характер, однако, обусловлены резкой остановкой тензометрической головки вместе с оснасткой и инструментом.

График 4 мало отличается от графика 3, однако, это сигнал имеющий нулевую постоянную составляющую. Столь малые отличия объясняются хорошей подстройкой переменного резистора в мостовой схеме включения тензометрических резисторов. Однако, такая подстройка не редко может производиться не точно или вообще не производится. Поэтому для получения результирующего сигнала используется алгоритм отброса постоянной составляющей. А именно, перевод сигнала в частотную область и вычисления амплитуды нулевой частоты.

Таким образом полученные данные становятся более читаемыми и пригодными к дальнейшему анализу, которы подразумевает под собой построении математической модели взаимодействия дискового режущего инструмента с ПСЛО, которая будет учитывать такие параметры как радиус закругления рабочей кромки и шаг резания.

	\FloatBarrier
	\providecommand*{\BibDash}{}                % В стилях ugost2008 отключаем использование тире как разделителя 
	\urlstyle{rm}                               % ссылки URL обычным шрифтом
	\ifdefmacro{\microtypesetup}{\microtypesetup{protrusion=false}}{} % не рекомендуется применять пакет микротипографики к автоматически генерируемому списку литературы
	\insertbiblio                          % Подключаем Bib-базы
	\ifdefmacro{\microtypesetup}{\microtypesetup{protrusion=true}}{}
	\urlstyle{tt}                               % возвращаем установки шрифта ссылок URL
\end{document} 
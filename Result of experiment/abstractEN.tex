\textbf{The urgency of the discussed issue.} To implement the government program <<Socio-economic development of the Arctic zone of the Russian Federation for the period until 2020>> approved by government decision \cite{PostRF} need to implement the strategy \cite{Strategi}. According to which the integration of the Arctic zone with the main regions of Russia is envisaged through: the exploitation of hydrocarbons, non-ferrous and precious metals; the formation of modern transport-logistics nodes and the core network of roads; development, reconstruction and modernization of the airport network.

This will erequire the maintenance, newly built and reconstructed roads and airfields during winter. The duration of which in some areas is more than 140 days a year. The most complex and important arrangements for the maintenance of road surfaces, for various purposes, aimed at the destruction of and removal of snow-ice formations (SIF). There are several ways of dealing with the SIF: chemical-mechanical; frictional; heat; mechanical. The latter method allows you to destroy and remove SIF from pavements without causing harm to the environment and to save on chemical reagents, fuel; to maintain the integrity of the roadway. It establishes mechanically the leader in the development and design of new workers on road cars.

However, there is a niche in which this method is not very effective, namely the removal of persistent snow-ice formations (PSIF). This is due to their physico-mechanical properties: density $\rho = 0,6 \div 0,9 \slantfrac{\text{g}}{\text{sm}^3}$.; the limit of compressive strength $\sigma = 2,5 \div 2,8$ MPa.; the thickness of the layer $h \le 100$ mm.; the temperature of the test medium $-2{}^\circ C\div-10{}^\circ C$. Existing working bodies or are not adapted to their destruction, or make it not effective. To improve performance and reduce energy consumption when you remove PSIF proposed the use of disk cutting tool \cite{GanjaDRI, GanjaPSLO, WorkOrgan}. However, with the use of disk cutting tool, the problem of creation of highly effective working bodies, at the design stage need to know about force paremeters, the value of which depends on many factors. Such as: cutting speed; the geometrical parameters of the instrument; ambient temperature and destructible material; value of the radius of curvature of the working edge. 

\textbf{The main aim of the study:} the main aim of the study the dependence of the force parameters, of the resistance to cutting of persistant snow-ice formations, on factors such as the radius of curvature of the working edge of the disk cutting tool and the cutting step. The work is a continuation of a series of experimental studies of the processes of interaction of the disk cutting tool with PSIF, for a few years, held at the Siberian Federal University.

\textbf{The methods used in the study:} When solving this problem, an integrated approach is applied, including: scientific analysis and generalization of the experience of previous studies; Experimental laboratory studies of the process of cutting ice with full-sized disk cutting tools with different radius of curvature of the working edge; mathematical and statistical processing of experimental results.

\textbf{The results:} Graphical dependences of the transient processes of cutting ice with different radius of the curvature of the working edge and the cutting step. The obtained dependences are analyzed and conclusions are made on the correctness of the experiment. Also, data were obtained for further analysis and development of a technique for calculating the cutting resistance force of the PSIF in the early stages of design, taking into account the influence of the radius of the curvature of the working edge, the disk cutting tool and the cutting step.
\usepackage{tabularx}

\renewcommand\Authfont{\small}
\renewcommand\Authsep{, \\}
\renewcommand\Authand{, \\}
\renewcommand\Authands{, \\}

%%% Кодировки и шрифты %%%
\setmainlanguage[babelshorthands=true]{russian}  % Язык по-умолчанию русский с поддержкой приятных команд пакета babel
\setotherlanguage{english}                       % Дополнительный язык = английский (в американской вариации по-умолчанию)
\setmonofont{Courier New}
\newfontfamily\cyrillicfonttt{Courier New}
\defaultfontfeatures{Ligatures=TeX,Mapping=tex-text}
\setmainfont{Times New Roman}
\newfontfamily\cyrillicfont{Times New Roman}
\setsansfont{Arial}
\newfontfamily\cyrillicfontsf{Arial}

%%% Размер шрифта %%%
\renewcommand{\tiny}{\fontsize{7}{8.4pt}\selectfont}
\renewcommand{\scriptsize}{\fontsize{9}{11pt}\selectfont}
\renewcommand{\footnotesize}{\fontsize{11}{13.6pt}\selectfont}
\renewcommand{\small}{\fontsize{12}{14.5pt}\selectfont}
\renewcommand{\normalsize}{\fontsize{14}{18pt}\selectfont}
\renewcommand{\large}{\fontsize{17}{20pt}\selectfont}
\renewcommand{\Large}{\fontsize{20}{25pt}\selectfont}
\renewcommand{\LARGE}{\fontsize{25}{30pt}\selectfont}


%%% Подписи %%%
\captionsetup{%
singlelinecheck=off,                % Многострочные подписи, например у таблиц
skip=2pt,                           % Вертикальная отбивка между подписью и содержимым рисунка или таблицы определяется ключом
justification=centering,            % Центрирование подписей, заданных командой \caption
}

%%% Рисунки %%%
%\DeclareCaptionLabelSeparator*{emdash}{~--- }             % (ГОСТ 2.105, 4.3.1)
\captionsetup[figure]{labelsep=period,position=bottom}

%%% Таблицы %%%
%\DeclareCaptionFormat{tablecaption}{\centering #1#2#3} 
\captionsetup[table]{labelsep=period,position=top}                 % разделитель идентификатора с номером от наименования
%\DeclareCaptionFormat{tablenocaption}{\tabcapalign #1\strut}    % Наименование таблицы отсутствует
\captionsetup[table]{singlelinecheck=off,position=top,skip=0pt}  % многострочные наименования и прочее
%\DeclareCaptionLabelFormat{table}{Таблица~#2}


\AtBeginDocument{%
    \setlength{\parindent}{12.5mm}                   % Абзацный отступ. Должен быть одинаковым по всему тексту и равен пяти знакам (ГОСТ Р 7.0.11-2011, 5.3.7).
}

%%% Списки %%%
% Используем короткое тире (endash) для ненумерованных списков (ГОСТ 2.105-95, пункт 4.1.7, требует дефиса, но так лучше смотрится)
\renewcommand{\labelitemi}{\normalfont\bfseries{--}}
\setlist{nosep,%                                    % Единый стиль для всех списков (пакет enumitem), без дополнительных интервалов.
    labelindent=\parindent,leftmargin=*%            % Каждый пункт, подпункт и перечисление записывают с абзацного отступа (ГОСТ 2.105-95, 4.1.8)
}

%%% Заголовки %%%
\makeatletter
%Section
\renewcommand{\section}{\@startsection{section}{1}%
{0pt}%отступ заголовка от левого поля
{1ex}% величина вертикального отступа, оставляемого перед заголовком.
{1ex}% величиа вертикального отступа после заголовка
{\normalfont\bfseries}}%стиль оформления заголовка
%subsection
\makeatother
\makeatletter
\renewcommand{\subsection}{\@startsection{subsection}{2}%
{0pt}%отступ заголовка от левого поля
{1ex}% величина вертикального отступа, оставляемого перед заголовком.
{1ex}% величиа вертикального отступа после заголовка
{\normalfont\bfseries}}%стиль оформления заголовка
\makeatother

\AtBeginDocument{%
   \def\postsection{~}% Один пробел без точки после номера секции
   \def\postsubsection{~}
}

%%% Авторы и места %%%
\renewcommand\Affilfont{\normalsize}
\renewcommand\Authfont{\normalsize}

%%% Заглавие статьи %%%
\makeatletter
    \def\@maketitle{%
  \newpage%
  \begin{flushleft}
    \noindent{УДК }\udk
  \end{flushleft}
  \begin{center}%
    {\normalsize\bfseries\MakeUppercase{\@title}}\par
    \vspace{1ex}
    \@author\par
  \end{center}}
\makeatother

%%% Формат блока аннотации %%%
\renewenvironment{abstract}{%
	  % \begin{center}
  	% 	\bfseries\abstractname\vspace{-.5em}\vspace{0pt}
  	% \end{center}
	  \list{}{%
	    \setlength{\listparindent}{\parindent} % Абзацный отступ аннотации
	  	\setlength{\itemindent }{\listparindent}
	    \setlength{\leftmargin}{0mm} % Поля аннотации
	    \setlength{\rightmargin}{\leftmargin}%
	    \setlength{\parsep}{\parskip}% Отступ ежду абзацами
	  }%
	  \item\relax}
{\endlist}

%%% Макет страницы %%%
\geometry{a4paper,top=25mm,bottom=25mm,left=15mm,right=15mm,nofoot,nomarginpar} %,showframe
\setlength{\topskip}{0pt}   %размер дополнительного верхнего поля

%%% Колонтитулы %%%
\pagestyle{empty} % нумерация страниц выкл.

%%% Интервалы %%%
%\DoubleSpacing*     % Двойной интервал
% \OnehalfSpacing*    % Полуторный интервал
%\OneSpacing*    % Одинарный интервал
%\setSpacing{1.42}   % Полуторный интервал, подобный Ворду (возможно, стоит включать вместе с предыдущей строкой)

%%% Выравнивание и переносы %%%
%% http://tex.stackexchange.com/questions/241343/what-is-the-meaning-of-fussy-sloppy-emergencystretch-tolerance-hbadness
%% http://www.latex-community.org/forum/viewtopic.php?p=70342#p70342
\tolerance 1414
\hbadness 1414
\emergencystretch 1.5em % В случае проблем регулировать в первую очередь
\hfuzz 0.3pt
\vfuzz \hfuzz
%\raggedbottom
% \sloppy                 % Избавляемся от переполнений
\clubpenalty=10000      % Запрещаем разрыв страницы после первой строки абзаца
\widowpenalty=10000     % Запрещаем разрыв страницы после последней строки абзаца